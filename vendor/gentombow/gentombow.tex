%# -*- ascii characters only -*-

\documentclass[a4paper]{article}
\usepackage{doc}
\makeatletter
%%% dangerous bend
\font\man=manfnt at 10pt
\def\dbend{\leavevmode\raise0pt\hbox{\man\char'177}}
\newenvironment{dangerous}{%
  \ifnum\@listdepth>\z@
    \GenericError{}{Do not use `dangerous' environment inside any list}{}{}
  \fi
  \par\addvspace\medskipamount
  \@tempdima=\dimexpr\textwidth-2zw\relax\small
  \divide\@tempdima by\dimexpr1zw\relax\@tempcnta=\@tempdima
  \leftskip=\dimexpr\textwidth-\@tempcnta zw\relax
    \@totalleftmargin\dimexpr\leftskip+0zw
    \linewidth=\dimexpr\@tempcnta zw-0zw
  \parindent1zw\noindent\kern-\leftskip\hbox to\leftskip{\dbend\hss}%
  \everypar{\everypar{}}\ignorespaces
}{\par\addvspace\medskipamount}
%%% misc
\newcommand{\Meta}[1]{$\langle$\mbox{}\textit{#1}\mbox{}$\rangle$}
%%%
\makeatother
\usepackage{longtable}
\usepackage[pdfbox]{gentombow}
%\usepackage{bxpapersize}
\GetFileInfo{gentombow.sty}
\title{Package \textsf{gentombow} \fileversion}
\author{Hironobu Yamashita}
\date{\filedate}
\begin{document}

\maketitle

It is convenient to print documents for final smaller paper sizes
on paper of the printer's standard physical paper size;
it allows printing close to the logical paper edge
and even outside the logical page.

This package \textsf{gentombow} provides a Japanese-style
crop marks (called `tombow' in Japanese) for trimming paper stacks,
and place the document page at the center of a larger physical
paper sheet. It automatically detects the page size
(\verb+\paperwidth+, \verb+\paperheight+) which is
going to be set by document class.
This document itself is a demonstration of this package,
with final A4 paper printed on B4 (JIS B4; not ISO B4!) paper.

A basic usage:
\begin{verbatim}
  \documentclass[a4paper]{article}
  \usepackage{gentombow}
  \usepackage{graphicx}
  \begin{document}
  The content
  \end{document}
\end{verbatim}

Note that this package does nothing about output paper size
specification; use one of the followings to do it.
\begin{itemize}
\item \textsf{graphicx} package
  with \texttt{setpagesize} feature added in 2016
  (supports all engines)
\item \textsf{bxpapersize} package
  (supports all engines; by Takayuki YATO)
\item \textsf{bounddvi} package
  (supports only DVI output mode; part of \textsf{gentombow} bundle)
\end{itemize}

This package is part of \textsf{gentombow} bundle:
\begin{verbatim}
  https://github.com/aminophen/gentombow
\end{verbatim}
Originally it was part of \textsf{platex-tools} bundle,
but it has been moved since March 2018.

\section{Basic Functions}

First, I introduce some basic functions of \textsf{gentombow} package.

\subsection{Automatic Determination of Output Size}

In this package,
the following printer's standard physical paper sizes are predefined;
A series (\texttt{a0}--\texttt{a10}),
B series (\texttt{b0}--\texttt{b10}),
C series (\texttt{c0}--\texttt{c10}),
\texttt{letter}, \texttt{legal} and \texttt{executive}.
In the current release, `B series' is JIS standard,
not ISO standard (this might be changed in the near future).
Also, some variations \texttt{a4var} and \texttt{b5var} are predefined.

When one of the above sizes (except \texttt{a0}, \texttt{b0}
and \texttt{c0}) is detected, the output size is
automatically determined along with the following rule:
\begin{itemize}
\item When A series detected:
        the output is going to be larger B series
\item When B/C series or
      \texttt{letter}, \texttt{legal}, \texttt{executive} detected:
        the output is going to be larger A series
\end{itemize}
The crop marks are automatically added.
The orientation (landscape/portrait) of the input paper is
also preserved.

The following list shows common examples:
\begin{longtable}[c]{cc}
  \hline
  Detected page size & Output size \\
  \hline
  a6 & b6 \\
  b6 & a5 \\
  a5 & b5 \\
  b5 & a4 \\
  a4 & b4 \\
  b4 & a3 \\
  a3 & b3 \\
  b3 & a2 \\
  \hline
  c6 & a5 \\
  c5 & a4 \\
  c4 & a3 \\
  c3 & a2 \\
  \hline
  letter    & a3 \\
  legal     & a3 \\
  executive & a4 \\
  \hline
\end{longtable}

\subsection*{When Automatic Determination Failed}

When the page size is different from any of the predefined sizes,
the page is placed with the surrounding 1~inch margins.
For example, when the page has $100\,\mathrm{mm}$ width and
$200\,\mathrm{mm}$ height, the output size is going to be
$100\,\mathrm{mm}+2\,\mathrm{in}$ width and
$200\,\mathrm{mm}+2\,\mathrm{in}$ height.

\subsection{Job Info Printing}

By default, the crop marks are printed with a banner, which shows
a job info like \makeatletter\texttt{\the\@bannertoken}\makeatother.
The format is similar to \verb+tombow+ option, which is
available in most of the common Japanese classes.
The default also reacts to \verb+tombo+ (without a job info) or
\verb+mentuke+ (do not print actual crop marks) class options.

\section{Package Options}

You can specify output size and/or disable job info printing
using package options.

\subsection{Explicit Specification of Output Size}

You can also force the output size using package option.
For example,
\begin{verbatim}
  \documentclass[a4paper]{article}
  \usepackage[tombow-a4]{gentombow}
  \begin{document}
  The content
  \end{document}
\end{verbatim}
forces the output size to be a3 (the automatic determination
`b4' is discarded). Available sizes are the same as
the predefined sizes, that is,
A series (\texttt{a0}--\texttt{a10}),
B series (\texttt{b0}--\texttt{b10}),
C series (\texttt{c0}--\texttt{c10}) and
\texttt{a4var}, \texttt{b5var},
\texttt{letter}, \texttt{legal}, \texttt{executive}.
The orientation (landscape/portrait) of the input paper is
also preserved again.

The option format is:
\Meta{crop mark format}\texttt{-}\Meta{output size}.
The crop mark format is one of the following:
\verb+tombow+ (default), \verb+tombo+ (without a job info),
\verb+mentuke+ (do not print actual crop marks).

\subsection{Disabling Job Info Printing}

When you specify the output size explicitly, the crop mark format
can be given at the same time, as described previously.
When the output size is automatically determined, you can disable
job info printing by \verb+notombowbanner+ option.

\section{Customization of Crop Mark}

The \textsf{pxgentombow} package provides some commands to
customize crop mark format.

\subsection{Banner content}

Using \verb+\settombowbanner+, you can set the banner content.
An example using \verb+\pdfcreationdate+ (pdf\TeX\ primitive)
\begin{verbatim}
  \documentclass[a4paper]{article}
  \usepackage{gentombow}
  \settombowbanner{\jobname\space (\pdfcreationdate)}
  \begin{document}
  The content
  \end{document}
\end{verbatim}
will result in
{\settombowbanner{\jobname\space (\pdfcreationdate)}%
 \makeatletter\texttt{\the\@bannertoken}\makeatother}.
The argument is an arbitrary token list.

\subsection{Banner font}

Using \verb+\settombowbannerfont+, you can change the font
with which the banner is printed.
\begin{verbatim}
  \settombowbannerfont{cmss10 at 9pt}
\end{verbatim}
This feature internally calls the \TeX\ primitive \verb+\font+,
and accepts all fonts supported by the engine. When Lua\TeX\ or
Xe\TeX\ is used, a native OpenType font can also be given.

\subsection{Crop mark line width}

By default, the line width of crop marks is 0.1~pt.
This can be change to 1~pt by \verb+\settombowwidth{1pt}+.
The argument is an arbitrary dimension.

\subsection{Bleed margin width}

By default, the bleed margin width is 3~mm.
This can be change to 5~mm by \verb+\settombowwidth{5mm}+.
The argument is an arbitrary dimension.

\subsection{Crop mark color}

Set the color of crop marks. Package \textsf{xcolor} (recommended) or
\textsf{xcolor} is required.
For example, \verb+\settombowcolor{\color[cmyk]{0,1,0,0}}+ sets magenta.

\section{Setting PDF page box (``digital tombow'')}

This is an optional driver-dependent feature.
When \textsf{gentombow} is required with the option \verb+pdfbox+,
following page boxes are emitted to the output PDF file.
The paper size (\verb+/MediaBox+) is also corrected.
\begin{itemize}
\item \verb+/TrimBox+: final paper size.
\item \verb+/BleedBox+: paper size plus bleed margin.
\item \verb+/CropBox+ and \verb+/ArtBox+ are not set.
\end{itemize}

\section{Additional Notes}

Here is some additional notes for use with several classes/packages.

\subsection{Note for \textsf{BXjscls} users}

When using \textsf{BXjscls} (by Takayuki YATO) with size option
other than 10pt, please add \verb+nomag+ or \verb+nomag*+ to the
class option. For example,
\begin{verbatim}
  \documentclass[a4paper,14pt]{bxjsarticle}
  \usepackage{gentombow}
\end{verbatim}
will not work as expected (\textsf{gentombow} throws an error
for safety).

\subsection{Note about Layout Settings}

You should not change \verb+\hoffset+ or \verb+\voffset+ to non-zero
value, for the purpose of layout settings. Such settings will
conflict with the feature of \textsf{gentombow} package.
To set page layout correctly, you should adjust \verb+\oddsidemargin+
or \verb+\topmargin+, or leave it to \textsf{geometry} package.

\section*{Change History}

\begin{itemize}
  \item 2017/12/17 v0.9 First CTAN release
  \item 2018/03/16 v0.9e Add \verb+\settombowbanner+,
                         \verb+\settombowbannerfont+ and
                         \verb+\settombowwidth+.
  \item 2018/05/17 v0.9g Add \verb+pdfbox+ option,
                         \verb+\settombowbleed+ and
                         \verb+\settombowcolor+.
  \item 2018/08/20 v0.9h Add \texttt{hagaki} size for paper detection,
                         improve patches for \textsf{pdfpages}.
  \item 2018/08/30 v0.9j Support \textsf{jsclasses}-style \verb+\mag+ employment
                         also with \verb+pdfbox+ option.
                         Fix a bug that \verb+\settombowbleed+ is not
                         considered by \verb+/TrimBox+.
\end{itemize}

\end{document}
